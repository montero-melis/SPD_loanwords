%This document has only the bare bone structure.
%Let's keep it for parts that are already in a good enough shape to be part of the final document.
%The work in progress will be done in 'ProjectProposal'

\documentclass[a4paper]{article}

\usepackage[T1]{fontenc} %words with accents but also other quite basic stuff, so it's good to have that!

\usepackage{fancyhdr} %Creates a fancy header to include course name, term etc. See "fancyhdr..." doc in LaTex documentation folder.

\usepackage{graphicx} %That's needed to insert external image files, e.g. JPEG. (I think...)
%\usepackage{ccaption} %Allows you to include legends in the figure environment with \legend{} command
%\usepackage{caption}  %Not sure what this does
%\usepackage{subcaption} %Possibly to have subheaders if you have two images under the same picture environment 

\usepackage{enumerate} % For instance lists where each item appears as a, b, c, etc.

%\usepackage{expex} %Numbered examples as often used by linguists. See "Bruno_introtex" in my LaTeX documentation folder for specific commands

\usepackage{multirow} %The easiest way to do multirow and multicolumn spanning in tables; more info found at:
% http://andrewjpage.com/index.php?/archives/43-Multirow-and-multicolumn-spanning-with-latex-tables.html

\usepackage{amsmath} %I think that's needed even for some quite basic mathematical type-setting

%% Bibliography:

%This bibliography style does not fit to any particular one (e.g. Harvard, APA), but is still quite good-looking:
\usepackage[style=authoryear-comp,maxnames=2,isbn=false,doi=false,firstinits=true, url=false]{biblatex}
%A slight variation with the options explained in the pdf file for biblatex package; yhe main difference is that 3 authors will be shown instead of directly "et al." if >3:
%\usepackage[style=authoryear-icomp, bibstyle=authoryear, minnames=3, isbn=false, url=false, doi=false]{biblatex}
\bibliography{bib} %Specify whatever the name of your bibliography here!
%Exported entries from Zotero have too many unwanted fields. This is the way to ged rid of them in the bibliography:
\AtEveryBibitem{% Clean up the bibtex rather than editing it
 \clearfield{date}
 \clearfield{eprint}
 \clearfield{isbn}
 \clearfield{issn}
 \clearfield{month}
 \clearfield{series}
 \clearfield{number}
 \clearfield{note}
 \ifentrytype{book}{}{% Remove publisher and editor except for books
 \clearlist{publisher}
 \clearname{editor}
 }
}

\title{The crosslinguistic intertextuality of loanwords}
\author{Vejdemo, Susanne \and Vandewinkel, Sigi \and Montero-Melis, Guillermo}

%Some handy shorthands for citing:
\newcommand{\pc}{\parencite}
\newcommand{\tc}{\textcite}



\begin{document}
\maketitle

%%%%%%%%%%%%%%
%%%%%%%%%%%%%%

\section{Purpose and aims}



%%%%%%%%%%%%%%
%%%%%%%%%%%%%%



\section{Survey of the field}


\subsection{Theory}


\subsubsection{Meaning (Walchli and Cysouw)}


%%%%%%%%%%%%%%


\subsubsection{Loanwords}


%%%%%%%%%%%%%%

\subsection{Method}


\subsubsection{Corpus linguistics}



\subsubsection{Experimental}







%%%%%%%%%%%%%%
%%%%%%%%%%%%%%


\section{Project description}


%%%%%%%%%%%%%%

\subsection{Phase I: Corpus phase}



%%%%%%%%%%%%%%

\subsection{Phase II: Experimental phase}




\subsection{Phase III: Compilation phase}


%%%%%%%%%%%%%%
%%%%%%%%%%%%%%


\section{Significance}


\subsection{Descriptive value}

\begin{itemize}
	\item The most detailed dictionary for (certain) loanwords.
	\item Database
	\item Semantic profiles
\end{itemize}


%%%%%%%%%%%%%%%%%% 



\subsection{Theoretical value}

\begin{itemize}
	\item What factors predict loanword integration?
	\item How much of a word's meaning potential is lost in language transfer?
	\item How is the semantic space shared between loanwords and their native equivalents?	
\end{itemize}


%%%%%%%%%%%%%%
%%%%%%%%%%%%%%


\section{Timeplan}


%%%%%%%%%%%%%%
%%%%%%%%%%%%%%


\section{Ethical considerations????}



\end{document}