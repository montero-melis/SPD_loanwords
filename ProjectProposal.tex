\documentclass[a4paper]{article}


%\usepackage[latin1]{inputenc}  %To use the ISO-8859-1 encoding
\usepackage[T1]{fontenc} %words with accents but also other quite basic stuff, so it's good to have that!

\usepackage{fancyhdr} %Creates a fancy header to include course name, term etc. See "fancyhdr..." doc in LaTex documentation folder.
\usepackage{graphicx} %That's needed to insert external image files, e.g. JPEG. (I think...)
%\usepackage{ccaption} %Allows you to include legends in the figure environment with \legend{} command
%\usepackage{caption}  %Not sure what this does
%\usepackage{subcaption} %Possibly to have subheaders if you have two images under the same picture environment 
\usepackage{enumerate} % For instance lists where each item appears as a, b, c, etc.
\usepackage{expex} %Numbered examples as often used by linguists. See "Bruno_introtex" in my LaTeX documentation folder for specific commands
\usepackage{multirow} %The easiest way to do multirow and multicolumn spanning in tables; more info found at:
% http://andrewjpage.com/index.php?/archives/43-Multirow-and-multicolumn-spanning-with-latex-tables.html
\usepackage{amsmath} %I think that's needed even for some quite basic mathematical type-setting

%% Bibliography:
%This bibliography style does not fit to any particular one (e.g. Harvard, APA), but is still quite good-looking:
\usepackage[style=authoryear-comp,maxnames=2,isbn=false,doi=false,firstinits=true, url=false]{biblatex}
%A slight variation with the options explained in the pdf file for biblatex package; yhe main difference is that 3 authors will be shown instead of directly "et al." if >3:
%\usepackage[style=authoryear-icomp, bibstyle=authoryear, minnames=3, isbn=false, url=false, doi=false]{biblatex}
\bibliography{bib} %Specify whatever the name of your bibliography here!
%Exported entries from Zotero have too many unwanted fields. This is the way to ged rid of them in the bibliography:
\AtEveryBibitem{% Clean up the bibtex rather than editing it
 \clearfield{date}
 \clearfield{eprint}
 \clearfield{isbn}
 \clearfield{issn}
 \clearfield{month}
 \clearfield{series}
 \clearfield{number}
  \clearfield{note}
 
 \ifentrytype{book}{}{% Remove publisher and editor except for books
  \clearlist{publisher}
  \clearname{editor}
 }
}

\usepackage[nodayofweek]{datetime} %A nice style for date, cf. document "datetime-formats" in LaTeX documentation folder; here I use \longdate, see below

\title{The crosslinguistic intertextuality of loanwords}
\author{Vejdemo, Susanne \and Vandewinkel, Sigi \and Montero-Melis, Guillermo}

\longdate % That's the specific style used with the datetime package (see above)
%The problem I've had is that I don't know how to set one specific date and have it written in the \longdate format...

%Some handy shorthands for citing:
\newcommand{\pc}{\parencite}
\newcommand{\tc}{\textcite}


\begin{document}
\maketitle

\thispagestyle{fancy} % Things have to go in this order for some reason
\lhead{\small Semantics, Pragmatics, and Discourse (FoSpr\aa k course) \\
 Stockholm University, Fall 2012 }


\section{Presentation and aims} 
 
The semantic meaning potential of words is to a large extent governed by their intertextual history of use (ref Linell, ref Traugott \& Dasher).
Loanwords are an interesting case, since they have a history in their source language, but lose some of their meaning potential when they are borrowed.
We wish to examine the following topics:
%
\begin{enumerate}[a.]
	\item \label{list_change} How much of a word's meaning changes when it is borrowed (e.g. the term `body guard', borrowed from English into Swedish), i.e how much of its meaning potential and history is lost?
	\item \label{list_share} After a loan word is established, how does it share the semantic space with already existing, seemingly synonymous native words (e.g. the Swedish term `livvakt', body guard.)
	\item How do \ref{list_change} and \ref{list_share} compare across three European languages, i.e. Swedish, Dutch, and Spanish?
\end{enumerate}
%
By using both quantitative experimental and corpus methods, as well as qualitative interview methods, we also wish to examine if there are marked discrepancies between the measurements of meaning and the subjective reported opinions about meanings of speakers.


%%%%%%%%%%%
%%%%%%%%%%%

\section{Theoretical background}


\subsection{Creating semantic profiles}
The increase of and access to computational power has made it possible to use large amounts of texts - corpora - to create semantic profiles for words. KOPTJEVSKAJATAMM-SAHLGREN has shown this for semantic investigations into temperature terms using a method known as Multidimensional Scaling (MDS). SIGSTRÖM has shown how semantic profiles can be made for the Swedish term \textit{helig} using the method Latent Semantic Analysis (LSA). 

Both MDS and LSA are based on comparing collocations of a words(see REF for an overview of collocations), that is occurrences of the words with a few words of context to the left and right, with a more semantic representation of the semantic relationships of words. Semantic representations are made by measuring how often all the words in a multimillion word text corpus co-occur with all other words in the corpus - the result is an n-dimensional space where words that co-occur often cluster together. (see DEERWEISTER, DUMAIS for general overviews of LSA, TURNEY-PANTEL http://www.jair.org/media/2934/live-2934-4846-jair.pdf for general overview of vector models for semantic processing).


\subsection{Comparing semantic profiles}
The semantic profiles we wish to compare can be represented as semantic vectors. Semantic vectors can be subtracted fromone another, leaving a semantic difference vector - a measure of the difference between the semantic profiles.(SIKSTRÖM SEMANTIC TESTS p19) Quantitatively, all words under investigation can be ranked according to how much they change: we can thus compare an English loanword with a native language equivalent. 

We also aim to develop a quantitative measurement of the semantic difference between the English word in the English corpora and the English loandword in the Swedish corpora. In addition, this comparison will also be done more qualitatively, by looking at which other words cluster with the target word. 


\subsection{What is meaning?}

Meaning can be approached in a variety of ways \cite{Koptjevskaja-Tamm2008}


\section{Susanne's draft}
HOW TO MEASURE MEANING CHANGE
As a word is borrowed fro one language to another, its meaning changes - the number and kind of referents it represents can grow or shrink, and its register (in which social setting it is appropriate to use) can also vary. This projects seeks to investigate this change - how large is it, and are there recurrent patterns in the kinds of changes that occur to loan words? 
This project aims to create both etic and emic definitions of the meaning, and meaning changes, of the loanwords. In an etic definition, the meaning of a word as defined as the set of its uses (Koptjevskaja-Tamm 2008), or the set of its situated instances (Evans 2009). In this, we follow the exemplar semantics research done by Wälchli \& Cysouw (2012), and will take a denotational approach to meaning where similarity in form will be assumed to represent similarity in meaning. In an emic definition, the commonalities behind the different uses are sought (Koptjevskaja-Tamm 2008), what Traugott \& Dasher 2002 call a "stabilized, institutionalized, and prototypical “magnetic center” that can be contextually interpreted in constrained ways" and which Linell (REF) refers to as the meaning potentials of a word.
The etic definitions will be investigated through the uses of the words in corpora (see section CL) and by acceptability judgments in the psycholinguistic research tradition (see section AJ). The emic definitions can aprtly be arrived of by a careful analysis of the commonalities in these two experiments, but the subconscious evaluations of the words by speakers will also be measured through psychological Semantic Differential experiments (see section SD).

%%%%%%%%%%%%
%%%%%%%%%%%

\subsection{Lexcial Borrowings}

\cite{Matras2009}

\section{Materials} 



%%%%%%%%%%%
%%%%%%%%%%%


\section{General method and phases}


%%%%%%%%%%%
%%%%%%%%%%%


\section{Expected outcome}


%%%%%%%%%%%
%%%%%%%%%%%

%\section{Sigis notes, to be built on and deleted}
%Some questions regarding lexical borrowings & hypotheses. These will prolly be too straightforward / fine-grained / detailed to be of any use as hypotheses.
%
%\begin{enumerate}[a.]
%	\item  \cite{Matras2009} writes on pp 172-173 when discussing integration of loans into the host language: ``We saw in Chapter 5 that bilinguals may choose to integrate some insertions but not to integrate others. This flexibility is largely limited, however, to active bilinguals; once a word has spread into the repertoire of monolinguals, it will tend to follow a fixed morpho-syntactic integration pattern.'' Does this entail hypotheses about the type of loans we expect in the three languages? The degree of incorporation? About differential incorporation (Swedes having the highest degree of bilingualism; Spaniards the least; Dutch intermediate)?
%	\item  Any hypotheses about the borrowing of cognates? Borrowing of compounds whose components have already been borrowed or are otherwise familiar (_bodyguard_ is perhaps easier to borrow if you have borrowed _body_). 
%	\item  Any hypotheses about the borrowing of English grammatical markers? Like plural markings? Dutch has a native -s plural; perhaps Spanish -os and -as can be counted as such too. Does this entail differences in the ease of incorporation of English flection? (e.g. Swedish _drinks_ vs Dutch _kids_ -- which feels less strange because of the native plural).
%	\item  Item the fourth
%\end{enumerate}

\section{Appendices}


\subsection{Swedish loanwords}


%\printbibliography

\end{document}
