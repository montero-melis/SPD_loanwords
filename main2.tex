\documentclass[a4paper]{article}

\usepackage{fancyhdr}
\usepackage{graphicx} 
\usepackage{enumerate}

\usepackage[a4paper]{geometry}
%\usepackage{fullpage}

\usepackage{url}

%\usepackage[T1]{fontenc} 
%\usepackage{ccaption} %Allows you to include legends in the figure environment with \legend{} command
%\usepackage{caption}  
%\usepackage{subcaption} 
%\usepackage{multirow} %The easiest way to do multirow and multicolumn spanning in tables; more info found at:
% http://andrewjpage.com/index.php?/archives/43-Multirow-and-multicolumn-spanning-with-latex-tables.html
%\usepackage{amsmath}1

%% Bibliography:
\usepackage[]{natbib}

\title{The crosslinguistic intertextuality of loanwords}
\author{Vejdemo, Susanne \and Vandewinkel, Sigi \and Montero-Melis, Guillermo}


\begin{document}
\maketitle
\thispagestyle{fancy} % Things have to go in this order for some reason
\lhead{\small Semantics, Pragmatics, and Discourse (FoSpr\aa k course) \\
 Stockholm University, Fall 2012 }
%%%%%%%%%%%%%%
%%%%%%%%%%%%%%

\section{Purpose and aims}
The semantic meaning potential of words is to a large extent governed by their intertextual history of use \citep{Linell2005,Traugott2001}.
Loanwords are an interesting case, since they have a history in their source language, but lose some of their meaning potential when they are borrowed.
There exists as yet no large scale investigation into the precise nature of this semantic change - thus this project has three ultimate aims: to create a freely available descriptive database with measurable meaning change data; 
to contribute to the theoretical knowledge on semantic change in loanword transfer;
and to develop a replicable methodology in order to assess meaning change over time.

\subsection{Research Questions}

We wish to examine the following topics:
%
\begin{enumerate}
	\item  How much of a word's meaning changes when it is borrowed (e.g. the term \emph{body guard}, borrowed from English into Swedish); that is, how much of its meaning potential and history is lost?
	\item  After a loan word is established, how does it share the semantic space with already existing, seemingly synonymous native words (e.g., with the Swedish term \emph{livvakt}, `body guard'.)
	\item How does the intertextual change compare across the two Germanic languages Dutch and Swedish?
	\item What factors determine whether a loanword is successfully integrated into a receiver language?
\end{enumerate}
%
By comparing results both from experimental and corpus methods, we also wish to examine if there are marked discrepancies between the measurements of meaning and the subjective reported opinions about meanings of speakers.


\subsection{Hypotheses}
\begin{enumerate}
	%Taken away hypothesis 1, since we don't believe this to be the case anyway.
	%\item Borrowings (e.g. \emph{body guard} as used in Swedish) will have the same semantic profiles as their (so stated) native equivalents (e.g. \emph{livvakt} in Swedish and their source language equivalents (e.g. \emph{body guard} as used in English).
	\item Borrowings (e.g., \emph{body guard} as used in Swedish) will not have the same semantic profiles as their (so stated) native equivalents (e.g., \emph{livvakt}) in Swedish or their source language equivalents (e.g. \emph{body guard} as used in English).
	They will either
	\begin{enumerate}[a)]
		\item differ in the registers they appear in, or
		\item differ in which referents they denote.
	\end{enumerate}
	\item Compounded borrowings will be easier integrated if their compound parts are also earlier borrowings - and the semantic profile of the compounded borrowing will be influenced by that of the compound part.
	\item The sociolinguistic background of speakers will to a significant extent affect their evaluation of loanwords.
\end{enumerate}
%%%%%%%%%%%%%%
%%%%%%%%%%%%%%



\section{Survey of the field}

%%%%%%%%%%%%%%

\subsection{Theory}

\subsubsection{Meaning}
\label{sect_meaning}

As a word is borrowed from one language to another, its meaning changes: the number and the kinds of referents it represents can grow or shrink, and its register value (the social settings in which it is appropriate to use) can also vary. This project seeks to investigate this change - how large is it, and are there recurrent patterns in the kinds of changes that are undergone by loanwords?

This project aims to create both etic and emic definitions of the meaning, and meaning changes, of the loanwords. 
In an etic definition, the meaning of a word is defined as the set of its uses \citep{Koptjevskaja-Tamm2008}, or the set of its situated instances \citep{Evans2009}. 
From this perspective the meaning of the Swedish word \emph{livvakt} or the Dutch word \emph{lijfwacht} is constituted by all the situated instances of its usage taken together
This may then be contrasted to, for instance, the meaning of the English word \emph{bodyguard}. 
In this, we follow the exemplar semantics research done by W{\"a}lchli and Cysouw, taking a denotational approach to meaning where similarity in form is assumed to correlate with similarity in meaning -- what the authors refer to as \emph{Isomorphism Hypothesis} \citep{Walchli2012}. 
In an emic definition, by contrast, the commonalities behind the different uses are sought \citep{Koptjevskaja-Tamm2008}, what \citet{Traugott2001} call a ``stabilized, institutionalized, and prototypical `magnetic center' that can be contextually interpreted in constrained ways'' and which \citet{Linell2005} refers to as the meaning potential of a word. Seen from this perspective, the meaning of Swedish \emph{livvakt} or Dutch \emph{lijfwacht} is most closely related to that of a central prototypical sense with fuzzy boundaries.

The etic definitions will be investigated through the uses of the words in corpora (see section~\ref{sect_corpus}) and by acceptability judgments and other tests in the psycholinguistic research tradition (see section~\ref{subsect_experim}). 
The emic definitions can partly be arrived at by a careful analysis of the commonalities in these two experiments, but the subconscious evaluations of the words by speakers will also be measured through psychological Semantic Differential experiments.

%%%%%%%%%%%%%%

%
\subsubsection{Loanwords}
\label{sect_loanwords}
Loanwords are the most straightforward way of studying languages in contact. 
They are highly visible, easily borrowable, and are subject to a measure of control by the speaker community \citep{Thomason2001}. 
One main reason for borrowings consists of filling referential gaps -- this is also the reason that the most common borrowings are nouns \citep[cf.][p.168]{Matras2009}.

Besides filling in referential gaps in the host language, the main reasons speakers borrow lexical items into their language have to do with various special effects: euphemisms, a need for trendiness and creativity \citep{Rebuck2002}; or for humorous effects, expressiveness or group identity \citep{Gottlieb2006,Wennberg2010}. 
This entails that borrowings are usually not quite synonymous with the native alternative(s) available in the host language: it seems that connotational or discourse-oriented distinctions suffice to warrant the incorporation of borrowed vocabulary alongside denotationally-synonymous native items. 
We expect this to show up clearly in the results of our LSA and experimental research: after all, it is commonly accepted that (near-) synonyms need not share the same antonyms \citep[cf.][]{Miller1990}. 

It is common for loanwords to become structurally integrated in the host language, phonetically as well as grammatically, with native phonemes and e.g. plurality markers substituting for the donor language's.
The degree of structural integration into the host language is often a correlate of the level of bilingualism prevalent in the speaker community.
Given that speakers of Swedish and Dutch may show significant difference in their familiarity with or fluency in English, we expect to see the effects of this in our data.
Furthermore, the typological and structural commonalities between our two languages should ensure a low threshold for borrowability and make incorporation feasible, minimizing interference stemming from typological incompatibility.

For the purposes of this project, we will adopt \citet[p.37]{Thomason1988}'s definition of borrowings: ``the incorporation of foreign features into a group's native language by speakers of that language'', since it stresses the agentive role of native speakers in the process \citep[see also][p.12]{Winford2003}. 
Operationally speaking, this means that we will not be dealing with hapax legoumena, single-speaker innovations or loanwords that are not commonly accepted by the speaker communities. 
Also excluded are lists of native alternatives to loanwords as solicited by popular-scientific publications, which have virtually no impact on the speech community at large. 
%SUS: WHAT DOES POPULAR SCIENTIFIC LIST DRIVE MEAN?
%SIG: I've clarified it. Better?
In short, only those loanwords that are in general usage, yet still engender prescriptivist pushback and that enjoy a relatively common native alternative will be considered. 
If existing lists of such pairs are any guide \citep[cf.][]{Koops2009,Universitet2012}, words that are part of what has been argued to be the core vocabulary of a language \citep[see][]{Swadesh1955} do not feature in them; instead, many are centred in the areas of technology, marketing and international relationships. 

Clearly the ``gap'' hypothesis of borrowing is insufficient to explain all or even most of the items on such word lists. 
By scrutinizing the details of pairs of anglicisms / native alternatives this study will propose precise measurements for various denotational and connotational aspects of their respective meanings -- specifically, denotational meaning loss or gain, and the typical patterns that occur when the semantic workload is divided and the shared semantic space is carved up.

%%%%%%%%%%%%%

\subsection{Method}


\subsubsection{Corpus linguistics}
%
\paragraph{Creating Semantic Profiles} \hspace{0pt} \\
The increase of and access to computational power has made it possible to use large amounts of texts -- corpora -- to create semantic profiles for words. 
\citet{Koptjevskaja-Tamm2013} have shown this for semantic investigations into temperature terms using a method known as Multidimensional Scaling (MDS). 
\citet{Sikstrom2012} use the method of Latent Semantic Analysis (LSA) to  establish semantic profiles for the Swedish term \textit{helig}. 

%G: Susanne, I've modified the explanation you had here; I think collocation is not the right term here, but 'distribution across contexts'; this is not just a picky thing but I believe bot to be quite distinct actually. We may talk more about that on some occasion...
Both MDS and LSA are based on comparing the distribution of words across contexts. That is, co-occurrences of different words in similar contexts are taken as evidence of a semantic relationships between them; very similar distributions entail very similar meanings. 
%Sus: Removed "(see REF for an overview of collocations)"
Semantic representations are made by measuring how often all the words in a multimillion word text corpus co-occur with all other words in the corpus; the result is an n-dimensional semantic space where words that co-occur often cluster together. 
See \citet{Berry1999, Turney2010} for a general overview of vector space models for semantic processing. 

This methodology has proven its worth in setting up semantic profiles in ways that allow for quantitative as well as qualitative analyses.
It is important to note here that we will not be reducing our data: one characteristic of this approach is that all data points are included in our analysis and none is discarded as an outlier; this makes it possible to capture both general trends and idiosyncrasies in the data  \citep[cf. also][p.678]{Bernardini2002,Walchli2012}.
% G: @Sus: I've taken away the not normalizing thing, because you do certain kinds of data transformation in LSA, but I guess the main message of not excluding certain data points still comes across.
%Sigi: I've reinserted the Bernardini /& Aston reference because it seems to me that we're referencing a single Wälchli/&Cysouw paper too often in the methodology secions.


\paragraph{Comparing Semantic Profiles} \hspace{0pt} \\
The semantic profiles we wish to compare are represented as vectors in a high-dimensional vector space. 
Degree of similarity or difference between words is measured using their cosine in the semantic space \citep{Deerwester1990}.
Quantitatively, all words under investigation can be ranked according to how similar they are to other words, yielding an enormous similarity matrix.
We also aim to develop a quantitative measure of the semantic difference between the English word in the English corpora and the English loanword in the Swedish and Dutch corpora. 
In addition, this comparison will also be done more qualitatively, by looking at which other words cluster with the target word. 

In order to get at the meaning of loanwords and their nearest synonyms in each language, we will use the theory and method of Latent Semantic Analysis, henceforth LSA.

\paragraph{Latent Semantic Analysis} \hspace{0pt} \\
LSA  \citep{Deerwester1990,Landauer1998,Dumais2004}  applies statistical computations to large corpora in order to build a semantic space.
The semantic space is derived by applying Singular Value Decomposition (SVD) to a large $ terms \times documents$ matrix, where the rows contain all the unique words that occur in the corpus, and the columns all the documents (i.e. contexts or texts) which form the database.
The resulting semantic space is the result of only retaining the first dimensions of the SVD decomposition, i.e. about a hundred dimensions as opposed to the many thousands in the original matrix%
\footnote{The choice of the dimension of the space is determined by the researcher, but typically it involves the 100 first dimensions or so extracted from a singular value decomposition. For the mathematical details see \citet{Deerwester1990}.}.

In this semantic space word meanings are represented as vectors.
The vector associated to each word summarizes the information of all the contexts it appears in, including what other words it typically co-occurs with.
The most intuitive way to understand similarity between words is that words with similar meaning are close to each other in this semantic space.
In fact, semantic differences are represented in terms of the cosine of the angle between any two vectors: the closer the cosine is to 1, the more similar in meaning the two words are; conversely, words with different meanings have a cosine value close to 0. 
Thus LSA offers a quantitative measure of \emph{how similar} two words actually are, which is exactly what we wish to determine for the loanwords and native alternatives in our samples.

Beyond inter-word similarity, LSA also allows us to test whether two words are used in similar types of texts across languages.
Thus, if we have external information about the texts of our database, e.g. that certain texts are, say, about computer technology and written by experts, we may see if a certain loanword is more often used in this type of texts or by a different set of users. 
Thus, there is a clear sociolinguistic aspect to our project.
Subsequently we can test if the same pattern applies in other languages having integrated the loanword, as well as in the donor language. 

\subsubsection{Experiments}
\label{subsect_experim}

We will develop a series of tests in order to evaluate the semantic and pragmatic intuitions of native speakers with different backgrounds.
The two key issues here are: a) the use of a valid and reliable test battery, and b) the use of appropriate sampling techniques in order to be able to draw generalizations on a wider population.

\paragraph{ Semantic and pragmatic tests}
\hspace{0pt} \\
It is useful to broadly differentiate between semantic and pragmatic tests, although they are necessarily related at a deeper level.
The former will test the inherent meaning of a particular word (section~\ref{sect_meaning}). 
The latter are concerned with appropriateness of use in certain contexts.

Semantic tests will include synonymy tests and semantic differential tests.
In a synonymy test a participant is asked to rate similarity among words.
In addition to pairwise ratings of similarity on a graded scale, we will use arrangement tasks, in which several words are placed on a surface (e.g., on a screen) according to their respective similarity \citep{Goldstone1994, Kriegeskorte2012}.
This will allow direct comparison to the similarity ratings obtained from LSA.
Semantic differential tests \citep{Osgood1957} will be the other tool to obtain a characterization of the position in the semantic space occupied by a certain word.
The framework of semantic differential also offers an appropriate tool to measure attitudes towards words, which we believe will show to be a determinant factor in the use of loanwords, as put forward in section~\ref{sect_loanwords} on loanwords.
Within semantic differential research it is well established that there are three cross-linguistically valid underlying factors that capture the attitude of speakers towards words: evaluation (`good--bad'), potency (`strong--weak'), and activity (`active--passive') \citep[see][]{Heise2010}.

As for the pragmatic tests, we will use acceptability ratings and production tasks in order to determine the appropriateness of using loanwords or their native equivalents in certain contexts.
Here we will directly draw on the database built up during the corpus phase.
Our stimuli now will not be single words but text fragments of different length.
From a pool of authentic contexts we will manipulate certain texts, replacing a loanword with its native equivalent or vice versa; in other cases the text will be the same as the original; the third possibility is to leave out the word in order to elicit a word from the participant.
All of the factors that might influence ratings will be crossed, and the design will be counterbalanced across participants.
The participant's task will then be either to judge acceptability of text fragments (acceptability ratings), or to provide the word that they deem most appropriate in a text were the original item has been left blank (production task).


\paragraph{Sampling technique}
\hspace{0pt} \\
Our goal is to ensure representative sampling of the population using web-based surveys.
All of the above tests can easily be run through the internet.
Moreover this method allows for massive collection of data without losing quality of data, and is rapidly gaining momentum in social research \citep{Denscombe2010}.
One of the main factors we wish to evaluate is how the sociolinguistic background of a person (e.g., their proficiency level in, and attitude towards, English) affects the outcome on the different tests.


%%%%%%%%%%%%%%%






%%%%%%%%%%%%%%
%%%%%%%%%%%%%%


\section{Project description}


%%%%%%%%%%%%%%

\subsection{Phase I: Corpus phase}
\label{sect_corpus}

The goal of the initial phase of the project is to produce semantic profiles of the source words in the source language, the source words as borrowings in the target languages and the native equivalences. 

The first step will be to create word specific literature overviews, using existing dictionaries and resources from other scholars who might have worked on the words. For Swedish, Svenska Akademien \citep{Akademien2012} and Spr\aa kbanken \citep{Universitet2012} provide such resources; for Dutch, large collections have been published by \cite{Koops2009} and \cite{Braamkolk2005}, which may be supplemented by large etymological dictionaries such as \citet{Sijs2001} and \citet{Sijs2005}.
For a small sample of the lexical items that might be included, see Appendix 1.

After this step, the initial list of potential concepts will have been reduced to a more manageable size. We wish to retain concepts that have a clear extension, and that are present in the culture of both language communities.

The next step will be to create semantic profiles using Latent Semantic Analysis, based on both source language and target language corpora. 
Two English corpora will be evaluated - the British National Corpus (BNC; 100m words) and the Corpus of contemporary American English (COCA; 450m words). 
For Dutch, both the INL corpus (38m words; \url{http://www.inl.nl/}) and perhaps the \textit{Corpus Gesproken Nederlands} \citep[Corpus of Spoken Dutch; 10m words;][]{Oostdijk2003} will be used. 
For Swedish, the various corpora available at Spr\aa kbanken (\url{http://www.spraakbanken.gu.se}) will be considered.

%%%%%%%%%%%%%%

\subsection{Phase II: Experimental phase}

The experimental phase has in itself three broad goals:
first, to \emph{validate} the results from the Corpus phase (cf. section~\ref{sect_corpus});
second, to \emph{explain} what factors determine that a certain loanword get integrated in the language, in the sense of being accepted by the speakers;
and finally to \emph{develop a methodology} consisting of different tests that can easily be replicated, making it possible to repeat measurements over time, and thus get an idea of how the studied phenomenon develops over time.

This phase will thus directly build on the results of the previous one, since the design of the experimental stimuli will be determined by the outcome of the corpus phase (cf. section~\ref{subsect_experim}).
At a practical level, we will first construct an internet-based platform to run the experiments on-line%
\footnote{There are different commercial solutions that offer appropriate interfaces to run this kind of survey, and other studies published in high-profile journals have used this approach \citep[e.g.][]{Scontras2012}.}.
Then we will collect the data using stratified sampling techniques to ensure representativity of the sample.
Finally the data will be analysed using standard statistical techniques to analyse semantic and attitude data.
Here we expect that the results will validate the findings from the first phase.

It is important to emphasize that this methodology can become a replicable standard, which can be exactly replicated at other moments in time, say after five years. 
It can also be easily adapted to incorporate new loanwords that find their way into a language.
Finally, given the degree of automatisation it will be adaptable to different languages if there exists a comparable-sized data base of texts for that language (note that the internet will provide such a corpus for relatively many of the world's languages).



%% Keep this commented just in case we want to revert to the list mode.
%The experimental phase has in itself three broad goals:
%\begin{enumerate}
%	\item Validation: validate the results from the Corpus phase (cf. section~\ref{sect_corpus}).
%	\item Explanation:
%	What factors determine that a certain loanword get integrated in the language, in the sense of being accepted by the speakers.
%	\begin{enumerate}
%		\item At the level of language, are there semantic or pragmatic factors influencing the level of integration?
%		\item At the individual level, is the level of knowledge of the donor language predictive of degree of acceptance of loanwords?
%	\end{enumerate}
%	\item Methodology: develop a methodology consisting of different tests that can easily be replicated, making it possible to repeat measurements over time, and thus get an idea of how the studied phenomenon develops over time. 
%\end{enumerate}
	


%%%%%%%%%%%%%%%

\subsection{Phase III: Compilation phase}
%Sigi will write -- has written, bitches!

The compilation phase of this project will focus on putting together a series of deliverables. These may be divided in three groups.

%writing up stuff & making available
In the first place, the compilation phase will be concerned with arranging and writing up all the data gathered in the previous steps, and finalizing the product for publication. One of the deliverables of this project is a monograph with extensive appendices, with among others a database accessible online. This database will be available to other researchers and lexicographers, and we believe that this database and the methodology outlined in the monograph will be of use in a variety of theoretically-oriented and practically-oriented projects.

%peer-reviewed publications
In the second place, the results arrived at will be written up as papers and submitted to high-profile peer-reviewed journals in the field of lexicology, semantics, pragmatics and typology; we project a further three or four deliverables for this project to be realised this way. Targets include \emph{Journal of Pragmatics}, \emph{Journal of Language Contact} and \emph{Journal of Semantics}; as well as lower-profile journals such as the electronic journal \emph{Lexis} and the annual \emph{Spr\aa k i Norden}.

%popular science publications
Furthermore, we will also submit a number of articles to popular-scientific publications in the language areas under study, notably the Swedish-language \emph{Spr\aa ktidningen} and the Dutch-language \emph{Onze Taal}.
Each of these regularly publish pieces on the influence of English, and we feel that they will be open to the findings of our rigorously empirically-based studies. We aim for at least three essays in these magazines. While this might not be a typical output for an academic research project, this will be a very concrete way to ensure that the practical benefits of this project are funneled out to non-academic language professionals, such as civil servants and translators.

%associated confs etc. I got these translator conferences from the list at http://www2.tolk.su.se/konfindx.html
In the third place, we will host a workshop on quantitative approaches to loanword studies at conferences traditionally centred on corpus linguistics and methodological issues in linguistics; an obvious target is \textsc{icame} or LREC. Further targets include business and translator conferences, such as the \emph{Nordic Translation Conference} (the 2013 installment will be held 4-6 April in Norwich, UK) and the Canadian Association for Translation Studies Conference (the next installment, held 3-5 June 2013, is themed \emph{Science in Translation}). 

It is projected that meeting these goals will require approximately one year, but portions of the research will of course be submitted for publication before this phase, as soon as they become available.

%%%%%%%%%%%%%%
%%%%%%%%%%%%%%


\section{Significance}
% Practical value: help dictionary makers. Get into prescriptivist magazines like spr\aa ktidningen \& onze taal. 
The academic significance of this project %is clear, and 
% Thought the previous phrase was a bit too "kaxig"
will be discussed in the two following sections. In addition to this, there are also several matters of  practical significance. One is to provide much needed input to the politicians and civil servants drafting language policies set to guard against the encroachment of English in Western Europe. Another is that the outcome of this research project can be of assistance to dictionary makers and lexicographers. Finally, we believe that a principled execution of our methodology will serve as an illustration of its values and an example to future linguists.

In order to reach this audience, it is a stated goal of the project to not only produce academic output, but also to publish in popular linguistic journals such as the Dutch-language \emph{Onze Taal} and the Swedish-language \emph{Spr\aa ktidningen}, as well as to attend business conferences for translators and to be active in its collaboration with language prescriptivist agencies in both countries. 
%Sigi says:"much better, this. Thanks" I have reverted the (Netherlands) and (Sweden) back to "Dutch-language" and "Swedish-language" -- let's not forget all those Flemish and Finnoswedish subscriptions to those magazines. Neither of these magazines operates in only one country.

\subsection{Descriptive value}
The descriptive value of the project is manifold. One outcome will be the first freely available database with detailed information on the use of loanwords in target language, source language, and, in addition, the native equivalence of the word in the target language. The published results will also include a very detailed dictionary of all the chosen loanwords, and their transition process. The semantic profiles created by LSA will be freely available, as will be the scripts that underlie their creation. 
%%%%%%%%%%%%%%%%%% 



\subsection{Theoretical value}
The theoretical value of the project lies in the discussion of the research questions, namely clearer answers to the following:
\begin{itemize}
	\item What factors predict loanword integration?
	\item How much of a word's meaning potential is lost in language transfer?
	\item How is the semantic space shared between loanwords and their native equivalents?	
\end{itemize}

The monograph we intend to accompany the database and the dictionary will clearly outline the methodology used in arriving at our results, making it easy for other researchers to adopt, or improve, our methods.


%%%%%%%%%%%%%%
%%%%%%%%%%%%%%


\section{Timeplan}
In terms of scheduling, we expect each of the three phases outlined in Section 3 to take up approximately one year. The process of writing up our results and preparing the final documents for publication will take longer than that, but since our results will come in incremental batches, interpreting and compiling them is more of a continuous process running in parallel with the other phases.

%%%%%%%%%%%%%%%
%%%%%%%%%%%%%%%
%
%
%%%No ethical considerations I'd say, especially if not all of the example proposals had such a section.
%%\section{Ethical considerations????}
%%Guillermo, you're the one writing about experiments on human subjects. Perhaps you could write this section?
%%Sigi suggests: perhaps mentioning that participants will have to sign documents allowing us to use the results but only for research \& if anonymous etc.
%%I'm also sure one of the project proposals on mondo has a section like this one.%

%%%%%%%%%%%%%%
%%%%%%%%%%%%%%

\bibliography{spdbib}{}
\bibliographystyle{apalike2}
%\bibliographystyle{plain}


\end{document}
