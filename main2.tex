\documentclass[a4paper]{article}

\usepackage{fancyhdr}
\usepackage{graphicx} 
\usepackage{enumerate}

%\usepackage[T1]{fontenc} 
%\usepackage{ccaption} %Allows you to include legends in the figure environment with \legend{} command
%\usepackage{caption}  
%\usepackage{subcaption} 
%\usepackage{multirow} %The easiest way to do multirow and multicolumn spanning in tables; more info found at:
% http://andrewjpage.com/index.php?/archives/43-Multirow-and-multicolumn-spanning-with-latex-tables.html
%\usepackage{amsmath}

%% Bibliography:
\usepackage[round]{natbib}

\title{The crosslinguistic intertextuality of loanwords}
\author{Vejdemo, Susanne \and Vandewinkel, Sigi \and Montero-Melis, Guillermo}


\begin{document}
\maketitle
\thispagestyle{fancy} % Things have to go in this order for some reason
\lhead{\small Semantics, Pragmatics, and Discourse (FoSpr\aa k course) \\
 Stockholm University, Fall 2012 }
%%%%%%%%%%%%%%
%%%%%%%%%%%%%%

\section{Purpose and aims}
The semantic meaning potential of words is to a large extent governed by their intertextual history of use (ref Linell, ref Traugott \& Dasher).
Loanwords are an interesting case, since they have a history in their source language, but lose some of their meaning potential when they are borrowed.
There exists as yet no large scale investigation into the precise nature of this semantic loss - thus this project has two ultimate aims: to create a freely available descriptive database with measurable meaning change data, and to contribute to the theoretical knowledge on semantic change in loanword transfer.

\subsection{Research Questions}
The semantic meaning potential of words is to a large extent governed by their intertextual history of use (ref Linell, ref Traugott \& Dasher).
Loanwords are an interesting case, since they have a history in their source language, but lose some of their meaning potential when they are borrowed.
We wish to examine the following topics:
%
\begin{enumerate}
	\item  How much of a word's meaning changes when it is borrowed (e.g. the term `body guard', borrowed from English into Swedish), i.e how much of its meaning potential and history is lost?
	\item  After a loan word is established, how does it share the semantic space with already existing, seemingly synonymous native words (e.g. the Swedish term `livvakt', body guard.)
	\item How do X and Y compare across three European languages, i.e. Swedish, Dutch, and Spanish?
	\item What factors determine whether a loanword is successfully integrated into a receiver language?
\end{enumerate}
%
By using both quantitative experimental and corpus methods, as well as qualitative interview methods, we also wish to examine if there are marked discrepancies between the measurements of meaning and the subjective reported opinions about meanings of speakers.


\subsection{Hypotheses}
\begin{enumerate}
	\item Borrowings will have the same semantic profiles as their (so stated) native equivalents.
	\item If they do not have the same semantic profiles, they will differ in which registers they appear in.
	\item If they do not have the same semantic profiles, they will differ in which referents they denote.
	\item Compounded borrowings will be more easier integrated if their compound parts are also earlier borrowings - and the semantic profile of the compounded borrowing will be influenced by that of the compound part.
	\item Likewise, the semantic profiles of borrowings will align with those of native cognates.
	
\end{enumerate}
%%%%%%%%%%%%%%
%%%%%%%%%%%%%%



\section{Survey of the field}


\subsection{Theory}


\subsubsection{Meaning (W\"alchli and Cysouw)}
As a word is borrowed from one language to another, its meaning changes - the number and kind of referents it represents can grow or shrink, and its register (in which social setting it is appropriate to use) can also vary. This projects seeks to investigate this change - how large is it, and are there recurrent patterns in the kinds of changes that occur to loan words?

\emph{This} project aims to create both etic and emic definitions of the meaning, and meaning changes, of the loanwords. In an etic definition, the meaning of a word as defined as the set of its uses (Koptjevskaja-Tamm 2008), or the set of its situated instances (Evans 2009). From this perspective the meaning of the Swedish word \emph{livvakt} or the Dutch word \emph{lijfwacht} is constituted by all the situated instances of its usage taken together; this may then be contrasted to, for instance, the meaning of the English loanword \emph{bodyguard}. In this, we follow the exemplar semantics research done by W\"alchli \& Cysouw (2012), and will take a denotational approach to meaning where similarity in form will be assumed to represent similarity in meaning. In an emic definition, by contrast, the commonalities behind the different uses are sought (Koptjevskaja-Tamm 2008), what Traugott \& Dasher 2002 call a "stabilized, institutionalized, and prototypical “magnetic center” that can be contextually interpreted in constrained ways" and which Linell (REF) refers to as the meaning potentials of a word. Seen from this perspective, the meaning of Swedish \emph{livvakt} or Dutch \emph{lijfwacht} is most closely related to that of a central prototypical sense with fuzzy boundaries.

The etic definitions will be investigated through the uses of the words in corpora (see section CL) and by acceptability judgments in the psycholinguistic research tradition (see section AJ). The emic definitions can aprtly be arrived of by a careful analysis of the commonalities in these two experiments, but the subconscious evaluations of the words by speakers will also be measured through psychological Semantic Differential experiments (see section SD).


%%%%%%%%%%%%%%


\subsubsection{Loanwords}
Loanwords are the most straightforward way of studying languages in contact. They're highly visible, easily borrowable, and are subject to a measure of control by the speaker community. One main reason for borrowings consists of filling referential gaps -- this is also the reason that the most common borrowings are nouns (cf. Matras 2009:168).

Besides filling in referential gaps in the host language, the main reason speakers borrow lexical items into their language have to do with various special effects: euphemisms, a need for trendiness and creativity (cf. Rebuck 2002); or for humorous effects, expressiveness or group identity (cf. Gottlieb 2006; Wennberg 2010). This entails that borrowings are usually not quite synonymous with the native alternative(s) available in the host language: it seems that connotational distinctions suffice to warrant the incorporation of borrowed vocabulary alongside denotationally-synonymous native items. We expect this to show up clearly in the results of our LSA research: after all, it is commonly accepted that (near-) synonyms need not share the same antonyms (cf. Miller et al 1990). 

It is common for loan words to become structurally integrated in the host language, phonetically as well as grammatically, with native phonemes and e.g. plurality markers substituting for the donor language's. The degree of structural integration into the host language is often a correlate of the level of bilingualism prevalent in the speaker community. Given that speakers of Swedish, Dutch and Spanish may show significant difference in their familiarity with or fluency in English (FOR REFERENCE SEE EMAIL), we expect to see differences here (FLESH OUT). 

We will not be dealing with true hapax legoumena, single-speaker innovations or not readily accepted loanwords; but only those words that are in general usage, yet still engender prescriptivist pushback. None of the words on the list are part of what has been argued to be the core vocabulary of a language (see Swadesh 1952): many are centred in the areas of technology, marketing and international relationships. Since the explicitly prescriptivist word lists include many single-morpheme native alternatives, the ``gap'' hypothesis of borrowing is clearly insufficient to explain them all. Furthermore, the typological and structural commonalities between our three languages should ensure a low threshold for borrowability and make incorporation feasible, minimizing interference stemming from typological incompatibility.
%Sus has doubts about this paragraph

%%%%%%%%%%%%%%

\subsection{Method}


\subsubsection{Corpus linguistics}

\paragraph{Creating Semantic Profiles} \hspace{0pt} \\
The increase of and access to computational power has made it possible to use large amounts of texts -- corpora -- to create semantic profiles for words. KOPTJEVSKAJATAMM-SAHLGREN has shown this for semantic investigations into temperature terms using a method known as Multidimensional Scaling (MDS). SIGSTR\"OM has shown how semantic profiles can be made for the Swedish term \textit{helig} using the method Latent Semantic Analysis (LSA). 

Both MDS and LSA are based on comparing collocations of a words(see REF for an overview of collocations), that is occurrences of the words with a few words of context to the left and right, with a more semantic representation of the semantic relationships of words. Semantic representations are made by measuring how often all the words in a multimillion word text corpus co-occur with all other words in the corpus - the result is an n-dimensional space where words that co-occur often cluster together. (see DEERWESTER, DUMAIS for general overviews of LSA, TURNEY-PANTEL http://www.jair.org/media/2934/live-2934-4846-jair.pdf for general overview of vector models for semantic processing).

It is important to note here that we will not be normalising our data: all data points will be included in our analysis and none will be discarded as outliers in order to bring out the generalities more strongly (cf. \cite{Walchli2012} 2012:678 and Bernardini and Aston 2002:293).

\paragraph{Comparing Semantic Profiles} \hspace{0pt} \\
The semantic profiles we wish to compare can be represented as semantic vectors. Semantic vectors can be subtracted from one another, leaving a semantic difference vector - a measure of the difference between the semantic profiles.(SIKSTRÖM SEMANTIC TESTS p19) Quantitatively, all words under investigation can be ranked according to how much they change: we can thus compare an English loanword with a native language equivalent. 

We also aim to develop a quantitative measurement of the semantic difference between the English word in the English corpora and the English loandword in the Swedish corpora. In addition, this comparison will also be done more qualitatively, by looking at which other words cluster with the target word. 
In order to get at the meaning of loanwords and their nearest synonyms in each language, we will use the theory and method of Latent Semantic Analysis, henceforth LSA CITE {Deerwester1990, Landauer1999, Dumais2004}.

\paragraph{Latent Semantic Analysis} \hspace{0pt} \\
LSA applies statistical computations to large corpora in order to build a semantic space.
The semantic space is derived by applying Singular Value Decomposition to a large $ terms \times documents$ matrix, where the rows contain all the unique words that occur in the corpus, and the columns the documents (i.e. contexts or texts) which form the database.

In this semantic space word meanings are represented as vectors in a high-dimensional space%
\footnote{The choice of the dimension of the space is determined by the researcher, but typically it involves the 100 first dimensions or so extracted from a singular value decomposition. For the mathematical details see CITE {Deerwester1990}.} where semantic differences are represented in terms of distance: the more different two senses are, the further apart they will be plotted from each other, and vice versa. Words with similar meaning are close to each other in this semantic space.
Thus LSA offers a quantitative measure of \emph{how similar} two words actually are, which is exactly what we wish to determine for each language.

Beyond inter-word similarity, LSA also allows  to test whether two words are used in similar types of texts across languages.
Thus, if we have external information about the texts of our database, e.g. that certain texts are, say, computer technology written by experts, we may see if a certain loanword is more often used in this type of texts or by a different set of users.
Subsequently we can test if the same pattern applies in the other languages, as well as in the donor language.


\subsubsection{Experimental}



%%%%%%%%%%%%%%%






%%%%%%%%%%%%%%
%%%%%%%%%%%%%%


\section{Project description}


%%%%%%%%%%%%%%

\subsection{Phase I: Corpus phase}
The goal of the initial phase of the project will be to have produced semantic profiles of the source words in the source language, the source words as borrowings in the target languages and the native equivalences. 

The first step will be to create word specific literature overviews, using existing dictionaries and resources from other scholars who might have worked on the words. For Swedish, XXX and YYY Spr\aa kr\aa det provide such information; for Dutch, large collections have been published by \cite{Koops2009} and \cite{Braamkolk2005}; for Spanish, ZZZ and AAA.

After this step, the initial list of potential concepts will have been reduced to a more manageable size. We wish to retain concepts that have a clear extension, that are present in all the culture of all three language communities.

The next step will be to create semantic profiles using Latent Semantic Analysis, based on both source language and target language corpora. Two English corpora will be evaluated - the British National Corpus (BNC; 100m words) and the Corpus of contemporary American English (COCA; 450m words), access to which has been generously provided by Mark Davies (REF). For Dutch, both the INL corpus (http://www.inl.nl/) and perhaps Corpus Gesproken Nederlands (Corpus of spoken Dutch ; http://lands.let.kun.nl/cgn/) will be used. For Swedish, the various corpora available at Spr\aa kbanken (http://www.spraakbanken.gu.se) will be considered.


%%%%%%%%%%%%%%

\subsection{Phase II: Experimental phase}
%Guillermo will write



\subsection{Phase III: Compilation phase}
%Sigi will write -- has written, bitches!

The compilation phase of this project will focus on putting together a series of deliverables. These may be divided in three groups.
%writing up stuff & making available
In the first place, the compilation phase will be concerned with arranging and writing up all the data gathered in the previous steps, and finalizing the product for publication. One of the deliverables of this project is a monograph with extensive appendices, with among others a database accessible online. This database will be available to other researchers and lexicographers, and we believe that this database and the methodology outlined in the monograph will be of use in a variety of theoretically-oriented and practically-oriented projects.

%peer-reviewed publications
In the second place, the results arrived at will be written up as papers and submitted to high-profile peer-reviewed journals in the field of lexicology, semantics, pragmatics and typology; we project a further three or four deliverables for this project to be realised this way. Targets include \emph{Journal of Pragmatics}, \emph{Journal of Language Contact}, \emph{Journal of Semantics}; as well as lower-profile journals such as the electronic journal \emph{Lexis} and the annual \emph{Spr\aa k i Norden}.

%popular science publications
Furthermore, we will also submit a number of articles to popular-scientific publications in the language areas under study, notably the Swedish-language \emph{Spr\aa ktidningen}, the Dutch-language \emph{Onze Taal} and the Spanish-language \emph{Lenguaje y Textos}. Each of these regularly publish pieces on the influence of English , and we feel that they will be open to the findings of our rigorously empirically-based studies. We aim for at least three essays in these magazines.

%associated confs etc. I got these translator conferences from the list at http://www2.tolk.su.se/konfindx.html
In the third place, we will host a workshop on quantitative approaches to loanword studies AT A LINGUISTICS CONFERENCE. ANY GOOD CONFERENCE SUGGESTIONS FOR THIS? Further targets include business and translator conferences, such as the \emph{Nordic Translation Conference} (the 2013 instalment will be held 4-6 April in Norwich, UK) and the Canadian Association for Translation Studies Conference (the next instalment, held 3-5 June 2013, is themed \emph{Science in Translation}). 

It is projected that meeting these goals will require approximately one year, but portions of the research will of course be submitted for publication before this phase, as soon as they become available.

%%%%%%%%%%%%%%
%%%%%%%%%%%%%%


\section{Significance}
% Practical value: help dictionary makers. Get into prescriptivist magazines like spr\aa ktidningen \& onze taal. 
The academic significance of this project is clear, and will be dealt with in the two following sections. In addition to this, there are also several matters of  practical significance. One is to provide much needed input to the politicians and civil servants drafting language policies set to guard against the encroachment of English in Western Europe. Another is that the outcome of this research project can help dictionary makers and lexicographers. SIGISUGGEST Finally, we believe that a principled execution of our methodology will serve as an illustration of its values and an example to future linguists END OF SIGISUGGEST

In order to reach this audience, it is a stated goal of the project to not only produce academic output, but also to publish in popular linguistic journals such as \emph{Onze Taal} (the Netherlands) and \emph{Spr\aa ktidningen} (Sweden), as well as to attend business conferences for translators and to be active in its collaboration with language prescriptivist agencies in both countries. 


\subsection{Descriptive value}
The descriptive value of the project is manifold. One outcome will be the first a freely available database with detailed information on the use of loanwords in target language, source language, and, in addition, the native equivalence of the word in the target language. The published results will also include a very detailed dictionary of all the chosen loanwords, and their transition process. The semantic profiles created by LSA will be freely available, as will be the scripts that underlie their creation. 
% Sigi suggests: "The monograph we intend to accompany the database and the dictionary will clearly outline the methodology used in arriving at our results, making it easy for other researchers to adopt and - or improve our methods."
%%%%%%%%%%%%%%%%%% 



\subsection{Theoretical value}
The theoretical value of the project lies in the discussion of the research questions, namely clearer answers to the following:
\begin{itemize}
	\item What factors predict loanword integration?
	\item How much of a word's meaning potential is lost in language transfer?
	\item How is the semantic space shared between loanwords and their native equivalents?	
\end{itemize}


%%%%%%%%%%%%%%
%%%%%%%%%%%%%%


\section{Timeplan}
%Sigi will write
In terms of scheduling, we expect each of the three phases outlined in Section 3 to take up approximately one year. The process of writing up our results and preparing the final documents for publication will take longer than that, but since our results will come in incremental batches, interpreting and compiling them is more of a continuous process running in parallel with the other phases.

%%%%%%%%%%%%%%
%%%%%%%%%%%%%%


\section{Ethical considerations????}
%Guillermo, you're the one writing about experiments on human subjects. Perhaps you could write this section?
%Sigi suggests: perhaps mentioning that participants will have to sign documents allowing us to use the results but only for research \& if anonymous etc.
%I'm also sure one of the project proposals on mondo has a section like this one.%

%%%%%%%%%%%%%%
%%%%%%%%%%%%%%

\bibliography{spdbib}{}
\bibliographystyle{apalike2}


\end{document}